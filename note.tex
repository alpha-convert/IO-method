\documentclass{article}
\usepackage[utf8]{inputenc}
%\usepackage[margin=30pt]{geometry}
\usepackage{amsmath}
%\usepackage{amsfonts}
\usepackage{amssymb}
\usepackage{graphicx}
%\usepackage{fourier}
\usepackage{amsmath}
\usepackage{amsfonts}
\usepackage{amsthm}
\usepackage{amssymb}
\usepackage{cmll}
%\usepackage{minted}
\usepackage{MnSymbol}
\usepackage{relsize}
\usepackage{enumerate}
\usepackage{proof}
\usepackage{mathpartir}
\usepackage{stmaryrd}
\usepackage{syntax}
%\usemintedstyle{bw}

\author{}
  
\newtheorem{theorem}{Theorem}[section]
\theoremstyle{definition}
\newtheorem{definition}{Definition}[section]
%\newtheorem{corollary}{Corollary}[section]





\newcommand{\isml}{\mintinline{sml}}
\newcommand{\msml}[1]{\text{\isml{#1}}}
\newcommand{\sml}{\isml}

\newcommand{\tytrans}[1]{\llbracket #1 \rrbracket}
\newcommand{\tertrans}[1]{\left\lVert#1\right\rVert}
\newcommand{\norm}[1]{\left\lVert#1\right\rVert}
\newcommand{\angles}[1]{\left\llangle #1 \right\rrangle}
\newcommand{\eval}{\downarrow}
\newcommand{\val}{\isml{val}}
\newcommand{\bdby}{\sqsubseteq}
\newcommand{\wknto}{\sqsubseteq}
\newcommand{\valbd}{\sqsubseteq_{\isml{val}}}
\newcommand{\subbd}{\sqsubseteq_{\isml{sub}}}
\newcommand{\subst}[2]{\left[#1 \middle/ #2 \right]}


\newcommand{\save}[3]{\ensuremath{\isml{save}^{#1}_{#2} \; #3}}
\newcommand{\disc}[2]{\ensuremath{\isml{disc}_{#1} \; #2}}
\newcommand{\wait}[2]{\ensuremath{\isml{wait}_{#1} \; #2}}
\newcommand{\tsfer}[6]{\ensuremath{\isml{transfer}_{#1} \, !^{#2}_{#3} \, #4 = #5 \; \isml{to} \; #6}}
\newcommand{\case}[5]{\ensuremath{\isml{case} \, (#1,\, #2. #3 \, , \, #4 . #5})}
\newcommand{\inl}[1]{\ensuremath{\isml{inl} \, #1}}
\newcommand{\inr}[1]{\ensuremath{\isml{inr} \, #1}}
\newcommand{\elist}{\ensuremath{\isml{[]}}}
\newcommand{\cons}[2]{\ensuremath{#1 \, :: \, #2}}
\newcommand{\listty}[1]{\ensuremath{\isml{[}#1\isml{]}}}
\newcommand{\delay}[1]{\ensuremath{\isml{delay} \, #1}}
\newcommand{\susp}[1]{\ensuremath{\isml{susp} \, #1}}
\newcommand{\force}[1]{\ensuremath{\isml{force} \, #1}}
\newcommand{\psplit}[4]{\ensuremath{\isml{split}(#1, \, #2.#3.#4)}}
\newcommand{\unit}{\ensuremath{\isml{()}}}
\newcommand{\N}{\ensuremath{\mathbb{N}}}
\newcommand{\nrec}[3]{\ensuremath{\isml{nrec}\left(#1,#2,#3\right)}}
\newcommand{\lrec}[3]{\ensuremath{\isml{lrec}\left(#1,#2,#3\right)}}
\newcommand{\inj}{\overline}
\newcommand{\tick}[1]{\ensuremath{\isml{tick} \; \, ; \; #1}}
\newcommand{\amp}{\ensuremath{\&}}
\newcommand{\amppair}[2]{\ensuremath{\langle #1,#2 \rangle}}
\newcommand{\scott}[1]{\ensuremath{\llbracket #1 \rrbracket}}
\newcommand{\snrec}{\ensuremath{\texttt{snrec}}}
\newcommand{\slrec}{\ensuremath{\texttt{slrec}}}
\newcommand{\scase}{\ensuremath{\texttt{scase}}}
\newcommand{\ret}[1]{\ensuremath{\texttt{ret} \, #1}}
\newcommand{\attach}[1]{\ensuremath{\texttt{attach} \, #1}}


\DeclareMathOperator{\wb}{WB}
\DeclareMathOperator{\wbc}{WBc}
\DeclareMathOperator{\Hom}{Hom}

\newcommand{\loli}{\multimap}
\newcommand{\tensor}{\otimes}
\newcommand{\proves}{\vdash}
\newcommand{\ang}{^{\circ}}

\newcommand{\curry}[1]{\ensuremath{\text{curry}\left(#1\right)}}
\newcommand{\const}[1]{\ensuremath{\text{const} \left(#1\right)}}

\newcommand{\ds}{\ensuremath{\$}}

\newcommand{\gens}{\Rightarrow}
\newcommand{\infers}{\uparrow}
\newcommand{\checks}{\downarrow}
\newcommand{\M}{\mathbb{M}}
\newcommand{\subty}{<:}
\newcommand{\subtynf}{<:_{\texttt{nf}}}

\newcommand{\fv}{\texttt{fv}}
\newcommand{\bv}{\texttt{bv}}

\title{Implementing Substructural Type Systems with the I/O Method}


\begin{document}

\iffalse
- Intro about substructural type systems
 - What are substructural type systems
 - Why do they force the context splitting?
 - Why is it hard to implement them?
   - (offload real impl to bidir summary paper)
- I/O Method
  - Declarative rules
  - I/O rules
- Unique determination
- Soundness
- Completeness
\fi


Substructural type systems are a tool for controlling the flow of resources and data around programs. At the most basic level, a substructural type system restricts the allowed use of variables in some way, in order to statically enforce some desired safety property. Some common examples of substructural type systems include \textit{linear} type systems, where every variable must be used exactly once, and \textit{affine} type systems where every variable must be used at least once. Affine and linear type systems both statically prevent a wide range of errors such as use-after-free bugs. Other (less common) examples include relevant and ordered type systems.

Substructural type systems work by making different so-called ``structural rules" of a type system inadmissable. For instance, linear logic disallows the use of the weakening and contraction rules-- since contraction allows for a variable in the context to be duplicated and weakening allows for a variable to be forgotten about, a program written in a type system which includes neither must use every variable exactly once.

The inadmissibility of these structural rules which describe the way the context must be manipulated is the heart of the added complexity in implementing substructural type systems. As a concrete example to illustrate this (which we will continue with for the remainder of this note), consider the following instance of the simply-typed lambda calculus with affine types.

\begin{mathpar}

\infer{\Gamma \vdash x : A}{x : A \in \Gamma}

\infer{\Gamma \vdash () : 1}{ }\\

\infer{\Gamma \vdash \lambda x : A. e : A \loli B}{\Gamma, x : A \vdash e : B}

\infer{\Gamma_1,\Gamma_2 \vdash e_1 \; e_2 : B}{\Gamma_1 \vdash e_1 : A \loli B \\ \Gamma_2 \vdash e_2 : A}

\end{mathpar}

The most interesting rule here is the application rule: note that the context is ``split" as two contexts $\Gamma_1,\Gamma_2$, and these two  contexts go towards typing different premises. This change over the standard rule (where the conclusion happens in context $\Gamma$, as do both premises) ensures that a contraction rule is inadmissible. 

Suppose we wanted to implement a type inference function for this language by writing a function \texttt{infer : ctx -> tm -> ty}. Most of the cases go through smoothly: the variable case is a lookup in the context, the lambda case adds a binding for $x : A$ in the context, and then recursively infers the type of $e$. But for function application, a naive solution gets stuck-- it's impossible to know how to split some input context to pass arguments to the two recursive calls. This is the crux of the problem.
\end{document}
